\documentclass{tufte-handout}

%\geometry{showframe}% for debugging purposes -- displays the margins

\usepackage{amsmath}
\usepackage[doipre={DOI:~}]{uri}

% Set up the images/graphics package
\usepackage{graphicx}
\setkeys{Gin}{width=\linewidth,totalheight=\textheight,keepaspectratio}
\graphicspath{{graphics/}}

\title{Time2Vec and the Nuances of Temporal Representation}
\author{
 Me et al.
}
%\date{24 January 2009}  % if the \date{} command is left out, the current date will be used

% The following package makes prettier tables.  We're all about the bling!
\usepackage{booktabs}

% The units package provides nice, non-stacked fractions and better spacing
% for units.
\usepackage{units}

% The fancyvrb package lets us customize the formatting of verbatim
% environments.  We use a slightly smaller font.
\usepackage{fancyvrb}
\fvset{fontsize=\normalsize}

% Small sections of multiple columns
\usepackage{multicol}

% Provides paragraphs of dummy text
\usepackage{lipsum}

% These commands are used to pretty-print LaTeX commands
\newcommand{\doccmd}[1]{\texttt{\textbackslash#1}}% command name -- adds backslash automatically
\newcommand{\docopt}[1]{\ensuremath{\langle}\textrm{\textit{#1}}\ensuremath{\rangle}}% optional command argument
\newcommand{\docarg}[1]{\textrm{\textit{#1}}}% (required) command argument
\newenvironment{docspec}{\begin{quote}\noindent}{\end{quote}}% command specification environment
\newcommand{\docenv}[1]{\textsf{#1}}% environment name
\newcommand{\docpkg}[1]{\texttt{#1}}% package name
\newcommand{\doccls}[1]{\texttt{#1}}% document class name
\newcommand{\docclsopt}[1]{\texttt{#1}}% document class option name

\begin{document}

\maketitle% this prints the handout title, author, and date
\vspace{12pt}

\begin{abstract}
Time2Vec, as described in the paper by Borealis et al.\cite{time2vec}, offers a vector representation of time, aiming to capture the characteristics of time in machine learning models. However, it can be susceptible to misinterpretation in various domains. This essay tries to explore the potential pitfalls of Time2Vec, especially when juxtaposed against the transformer architecture.
\end{abstract}

\section{Strengths Revisited}

\begin{enumerate}
\item Time2Vec's ability to encapsulate the progression, periodicity, and scale of time is of great interest. The representation, while being versatile across datasets, captures time's essence in a manner that's more sophisticated than mere timestamps.

\item The transformer architecture, though originally crafted for NLP, has been employed in numerical timeseries classification, as seen in various articles (show articles). Time2Vec's potential as a positional encoding mechanism in such architectures underscores its versatility.
\end{enumerate}

\section{Concerns and Clarifications}

\begin{enumerate}
\item Many articles seem to use timeseries values (like prices or sales) as inputs to Time2Vec, rather than the actual temporal information (i.e., a time index \( t=0,1,2,3...\)). This approach seems more like an enrichment of feature representation rather than true positional encoding. Even the original Word2Vec paper\cite{word2vec}, which inspired Time2Vec, hints at its design being more aligned with time-related features.

\item Terms like "positional encoding" have specific meanings. If Time2Vec is used to encode features rather than positions, it leads to potential confusion. Such confusion can compound when architectures are layered and interwoven in complex models.

\item Even if Time2Vec were applied on a time index, it should likely be a global time index created before batching or windowing the data. This is because the local time index within a batch might be arbitrary, potentially introducing inconsistencies from a theoretical point of view.
\end{enumerate}

\section{Conclusion}

Time2Vec, in its essence, presents an alternative approach to repersenting time. However, its application needs careful consideration. Misinterpretations, especially in areas as intricate as positional encoding, can lead to flawed models and misguided conclusions. A more rigorous mathematical framework and clearer guidelines for Time2Vec's application might be required.


\section{Acknowledgements}\label{sec:support}

This report was typeset using \LaTeX, originally developed by Leslie Lamport and based on Donald Knuth's \TeX. A template that can be used to format documents with this look and feel has been released under the permissive \href{http://www.apache.org/licenses/LICENSE-2.0}{\textsc{Apache License 2.0}}, and can be found online at \url{https://tufte-latex.github.io/tufte-latex/}.

\bibliography{sample-handout}
\bibliographystyle{plainnat}

\end{document}